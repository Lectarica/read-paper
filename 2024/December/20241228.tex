\documentclass[a4paper,12pt]{jarticle}
\usepackage[dvipdfmx]{graphicx}
\usepackage{amsmath,amssymb,amsfonts}
\usepackage{geometry}
\geometry{top=25mm, bottom=25mm, left=20mm, right=20mm}
\usepackage{hyperref}

\title{Superconductivty without inversion symmetry: MnSi versus CePt3Si}
\author{P. Frigeri, D.F. Agterberg, A. Koga, and M. Sigrist}
\date{today}

\linespread{0.9}
\begin{document}

\maketitle

\section*{アブストの翻訳}
空間反転対称性が存在しない物質における超伝導を研究した。我々は、一般に考えられているのとは異なり、こうした系においてスピン三重項対形成が完全には排除されないことを示す。さらに、スピン一重項と三重項の両方の対形成におけるパラ磁性による制限(パラメトリック・リミティング)を解析する。空間反転対称性の欠如は、スピン一重項の対形成に対するパラ磁性の制限を弱めることがわかった。これらの結果を MnSi および CePt3Si へ適用した。

\section*{研究背景・問題意識}
\begin{itemize}
    \item 本来時間反転対称性が破れると、超伝導は抑制される。また、反転対称性も電子状態の縮退を得るために必要であると指摘されている。しかし反転対称性を欠くCePt3Siでは超伝導が発見されている。
\end{itemize}

\section*{研究目的}
\begin{itemize}
    \item 反転中心を欠きつつも時間反転対称性を保つ物質について、スピントリプレットが存在するか?
    \item パラ磁性による制限の問題を明らかにしたい
\end{itemize}

\section*{モデルと手法}
\begin{itemize}
    \item 
\end{itemize}

\begin{equation}
    H
    = \sum_{k,s} \xi_k \, c_{ks}^\dagger \, c_{ks}
    + \frac{1}{2} \sum_{k,k'} \sum_{s,s'} V_{k,k'}
    \, c_{ks}^\dagger \, c_{-k\,s'}^\dagger 
    \, c_{-k'\,s'} \, c_{k'\,s}
    \,.
    \tag{1}
\end{equation}

\section*{結果}
\begin{itemize}
    \item 
\end{itemize}

\end{document}
    